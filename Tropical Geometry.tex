\documentclass[russian]{lecture-notes}
\usepackage{dsfont}
\usepackage{amsmath}
\usepackage{amssymb}


\DeclareMathOperator{\trk}{trk}
\DeclareMathOperator{\Krk}{Krk}
\DeclareMathOperator{\Trop}{Trop}
\DeclareMathOperator{\Brk}{Brk}
\DeclareMathOperator{\trdeg}{trdeg}

\title{Введение в тропическую математику}
\author{Дмитрий Зайков}
\date{20.07.2019}


\begin{document}
	\section{Тропическая геометрия нейронных сетей}
	
	В разделе 5 нейронные сети определяются с помощью тропической алгебры, что позволяет нам изучать их с помощью тропической алгебраической геометрии. Мы покажем, что граница принятий решений нейронной сети ~--- это подмножество тропической гиперповерхности, соответствующего тропического полинома(Раздел 6.1). Мы увидим, что в некотором смысле, зонотопы образуют геомтрические строительные блоки для нейронных сетей (Раздел 6.2). Затем мы докажем, что геометрия функции, представленной нейронной сетью, становится значиттельно более сложной с увеличением ее количества слоёв.
	\subsection{Границы решений нейронной сети}
	
	Мы будем использовать тропическую геометрию и идеи из Раздела 5 для изучения границ решений нейронных сетей, фокусируясь на случае классификации двух категорий для ясности. Как объяснено в Разделе 4, нейронная сеть $\nu : \mathbf{R}^d \rightarrow \mathbf{R}^p$ вместе с выбором функции оценки $s: \mathbf{R}^p \rightarrow \mathbf{R}$ дают нам классификатор. 

\end{document}