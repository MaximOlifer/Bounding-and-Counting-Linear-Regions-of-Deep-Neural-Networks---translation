\documentclass[russian]{lecture-notes}
\usepackage{dsfont}
\usepackage{amsmath}
\usepackage{amssymb}


\DeclareMathOperator{\trk}{trk}
\DeclareMathOperator{\Krk}{Krk}
\DeclareMathOperator{\Trop}{Trop}
\DeclareMathOperator{\Brk}{Brk}
\DeclareMathOperator{\trdeg}{trdeg}

\title{Гропическая геометрия глубоких нейронных сетей}
\author{Дмитрий Зайков, Олифер Максим}
\date{20.10.2019}


\begin{document}
	\begin{abstract}
		Впервые мы установили связь между нейронными сетями без обратной связи с ReLU активациями и тропической геометрией ~--- мы показываем, что семейство таких нейронных сетей эквивалентна семейству тропических рациональных карт. Среди прочего, мы вывели, что ReLU нейросеть без обратной связи с одним скрытым слоем может быть характеризована зонотопами, которые могут послужить блоками для строительства более глубоких сетей; мы связываем определение границ таких нейронных сетей с тропическими гиперповерхностями, главный объект изучения в тропической геометрии; и мы доказываем, что линейные области таких нейронных сетей соответствуют вершинам политопов, связанных с тропическими рациональными функциями. Вывод из нашей тропической формулировки такой, что более глубокая нейронная сеть экспоненциально более выразительная, чем менее глубокая.
	\end{abstract}
	
	\section{Тропическая геометрия нейронных сетей}
	
	В разделе 5 нейронные сети определяются с помощью тропической алгебры, что позволяет нам изучать их с помощью тропической алгебраической геометрии. Мы покажем, что граница принятий решений нейронной сети ~--- это подмножество тропической гиперповерхности, соответствующего тропического полинома(Раздел 6.1). Мы увидим, что в некотором смысле, зонотопы образуют геомтрические строительные блоки для нейронных сетей (Раздел 6.2). Затем мы докажем, что геометрия функции, представленной нейронной сетью, становится значиттельно более сложной с увеличением ее количества слоёв.
	\subsection{Границы решений нейронной сети}
	
	Мы будем использовать тропическую геометрию и идеи из Раздела 5 для изучения границ решений нейронных сетей, фокусируясь на случае классификации двух категорий для ясности. Как объяснено в Разделе 4, нейронная сеть $\nu : \mathbf{R}^d \rightarrow \mathbf{R}^p$ вместе с выбором функции оценки $s: \mathbf{R}^p \rightarrow \mathbf{R}$ дают нам классификатор. 

\end{document}