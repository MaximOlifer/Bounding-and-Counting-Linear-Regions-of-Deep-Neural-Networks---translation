\documentclass[russian]{lecture-notes}
\usepackage{dsfont}
\usepackage{amsmath}
\usepackage{amssymb}


\DeclareMathOperator{\trk}{trk}
\DeclareMathOperator{\Krk}{Krk}
\DeclareMathOperator{\Trop}{Trop}
\DeclareMathOperator{\Brk}{Brk}
\DeclareMathOperator{\trdeg}{trdeg}

\title{Гропическая геометрия глубоких нейронных сетей}
\author{Дмитрий Зайков, Олифер Максим}
\date{20.10.2019}


\begin{document}
	\begin{abstract}
		Впервые мы установили связь между нейронными сетями без обратной связи с ReLU активациями и тропической геометрией ~--- мы показываем, что семейство таких нейронных сетей эквивалентна семейству тропических рациональных карт. Среди прочего, мы вывели, что ReLU нейросеть без обратной связи с одним скрытым слоем может быть характеризована зонотопами, которые могут послужить блоками для строительства более глубоких сетей; мы связываем определение границ таких нейронных сетей с тропическими гиперповерхностями, главный объект изучения в тропической геометрии; и мы доказываем, что линейные области таких нейронных сетей соответствуют вершинам политопов, связанных с тропическими рациональными функциями. Вывод из нашей тропической формулировки такой, что более глубокая нейронная сеть экспоненциально более выразительная, чем менее глубокая.
	\end{abstract}

	\section{Введение}
	
	Глубокие нейронные сети недавно получили много внимания за их огромный успех в ряде приложений из многих областей искуственного интеллекта, компьютерного зрения, распознавания речи и генерации естественного языка. (LeCun et al., 2015; Hinton et al., 2012; Krizhevsky et al., 2012; Bahdanau et al., 2014; Kalchbrenner \& Blunsom, 2013). Тем не менее, также известно, что наше теоретическое понимание их эффективности остается неполным.
	
	Было несколько попыток анализа глубоких нейронных сетей с разных перспектив. Особенно ранние работы показали, что глубокие архитектуры могут использовать параметры более эффективно и требуют экспоненциально меньше параметров, чтобы выражать определенные семейства функций, чем менее неглубокие архитектуры (Delalleau \& Bengio, 2011; Bengio \& Delaleau, 2011; Montufar et al., 2014; Eldan \& Shamir, 2016; Poole et al., 2016; Arora et al., 2018). Недавняя работа (Zhang et al., 2016) показала, что несколько успешных нейронных сетей обладают большой репрезентативностью и могут легко разбивать случайные данные. Однако они также хорошо обобщают данные, которые они не видели, во время тренировочного этапа, поэтому можно предположить, что такие сети могут иметь некоторую встроенную регуляризацию. Традиционные меры сложности, такие как размерность Вапника-Червоненкиса и Rademacher complexity, не могут объяснять данный феномен. Понимание данной встроенной регуляризации, порождающей обобщающую мощь глубоких нейронных сетей остается проблемой.
	
	Цель нашей работы - установить связи между нейронными сетями и тропической геометрией в надежде, что она прольет свет на работу глубоких нейронных сетей. Тропическая геометрия это новая область алгебраической геометрии, пережившей взрывной рост за последнюю декаду, но остающейся относительно беззвестной за пределами чистой математики. Мы сконцентрируемся на нейронных сетях без обратной связи с ReLU и покажем, что они являются аналогаим рациональных функций, т.е. отношением двух многомерных многочленов f, g для переменных $x_1,...,x_d$.
	
	
	\section{Тропическая геометрия нейронных сетей}
	
	В разделе 5 нейронные сети определяются с помощью тропической алгебры, что позволяет нам изучать их с помощью тропической алгебраической геометрии. Мы покажем, что граница принятий решений нейронной сети ~--- это подмножество тропической гиперповерхности, соответствующего тропического полинома(Раздел 6.1). Мы увидим, что в некотором смысле, зонотопы образуют геомтрические строительные блоки для нейронных сетей (Раздел 6.2). Затем мы докажем, что геометрия функции, представленной нейронной сетью, становится значиттельно более сложной с увеличением ее количества слоёв.
	\subsection{Границы решений нейронной сети}
	
	Мы будем использовать тропическую геометрию и идеи из Раздела 5 для изучения границ решений нейронных сетей, фокусируясь на случае классификации двух категорий для ясности. Как объяснено в Разделе 4, нейронная сеть $\nu : \mathbf{R}^d \rightarrow \mathbf{R}^p$ вместе с выбором функции оценки $s: \mathbf{R}^p \rightarrow \mathbf{R}$ дают нам классификатор. 

\end{document}
