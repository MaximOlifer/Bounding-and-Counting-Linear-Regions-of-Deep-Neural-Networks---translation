\documentclass[russian]{lecture-notes}
\usepackage{dsfont}
\usepackage{amsmath}
\usepackage{amssymb}


\DeclareMathOperator{\trk}{trk}
\DeclareMathOperator{\Krk}{Krk}
\DeclareMathOperator{\Trop}{Trop}
\DeclareMathOperator{\Brk}{Brk}
\DeclareMathOperator{\trdeg}{trdeg}

\title{Гропическая геометрия глубоких нейронных сетей}
\author{Дмитрий Зайков, Олифер Максим}
\date{20.10.2019}


\begin{document}
	\begin{abstract}
		Впервые мы установили связь между нейронными сетями без обратной связи с ReLU активациями и тропической геометрией ~--- мы показываем, что семейство таких нейронных сетей эквивалентна семейству тропических рациональных карт. Среди прочего, мы вывели, что ReLU нейросеть без обратной связи с одним скрытым слоем может быть характеризована зонотопами, которые могут послужить блоками для строительства более глубоких сетей; мы связываем определение границ таких нейронных сетей с тропическими гиперповерхностями, главный объект изучения в тропической геометрии; и мы доказываем, что линейные области таких нейронных сетей соответствуют вершинам политопов, связанных с тропическими рациональными функциями. Вывод из нашей тропической формулировки такой, что более глубокая нейронная сеть экспоненциально лучше выражает, чем менее глубокая.
	\end{abstract}

	\section{Введение}
	
	Глубокие нейронные сети недавно получили много внимания за их огромный успех в ряде приложений из многих областей искуственного интеллекта, компьютерного зрения, распознавания речи и генерации естественного языка. (LeCun et al., 2015; Hinton et al., 2012; Krizhevsky et al., 2012; Bahdanau et al., 2014; Kalchbrenner \& Blunsom, 2013). Тем не менее, также известно, что наше теоретическое понимание их эффективности остается неполным.
	
	Было несколько попыток анализа глубоких нейронных сетей с разных перспектив. Особенно ранние работы показали, что глубокие архитектуры могут использовать параметры более эффективно и требуют экспоненциально меньше параметров, чтобы выражать определенные семейства функций, чем менее неглубокие архитектуры (Delalleau \& Bengio, 2011; Bengio \& Delaleau, 2011; Montufar et al., 2014; Eldan \& Shamir, 2016; Poole et al., 2016; Arora et al., 2018). Недавняя работа (Zhang et al., 2016) показала, что несколько успешных нейронных сетей обладают большой репрезентативностью и могут легко разбивать случайные данные. Однако они также хорошо обобщают данные, которые они не видели, во время тренировочного этапа, поэтому можно предположить, что такие сети могут иметь некоторую встроенную регуляризацию. Традиционные меры сложности, такие как размерность Вапника-Червоненкиса и Rademacher complexity, не могут объяснять данный феномен. Понимание данной встроенной регуляризации, порождающей обобщающую мощь глубоких нейронных сетей остается проблемой.
	
	Цель нашей работы - установить связи между нейронными сетями и тропической геометрией в надежде, что она прольет свет на работу глубоких нейронных сетей. Тропическая геометрия это новая область алгебраической геометрии, пережившей взрывной рост за последнюю декаду, но остающейся относительно беззвестной за пределами чистой математики. Мы сконцентрируемся на нейронных сетях без обратной связи с ReLU и покажем, что они являются аналогаим рациональных функций, т.е. отношением двух многомерных многочленов f, g для переменных $x_1,...,x_d$,
	
	\begin{equation*}
	\frac{f(x_1,...,x_d)}{g(x_1,...,x_d)},
	\end{equation*}
	
	в тропической геометрии. Для стандартных и тригонометрических полиномов известно, что рациональная аппроксимация --- это приближение функции отношением двух полиномов вместо одного полинома --- значительно улучшает качество аппроксимации без повышения степени. Это дает аналогию: нейронная сеть с ReLU это тропическое отношение двух тропических многочленов, т.е. тропическая рациональная функция. Точнее, если мы рассмотрим нейронную сеть как функцию $v : R^d \Rightarrow R^p, x=(x_1,...,x_d) \implies ((v_1(x)),...,v_p(x))$, где каждая v это тропическое пространство, т.е. каждая v это тропическая рациональная функция. Фактически, мы покажем, что
	
	
	\textit{семейство функций, представленных нейронной сетью без обратной связи с ReLU и целочисленными весами это в точности семейство тропических рациональных пространств.}
	
	Отсюда немедленно следует, что на данном семействе существует полуполе. Что более важно, это устанавливает мост между нейронными сетями и тропической геометрией, что позволяит нам рассмотреть нейронные сети как хорошо изученные тропические объекты. Это знание помогает нам связать ближе гриницы между линейными областями в нейронных сетях с тропическими гиперповерхностями и тем самым способствовать в определентю границ нейронной сети для проблемы классификации как тропических гиперповерхностей. Более того, количество линейных областей, что является мерой сложности нейронной сети (Montufar et al., 2014; Raghu et al., 2017; Arora
	et al., 2018), может быть ограничено количеством вершин политопа, связанного с тропическим представлением нейронной сети. Наконец, нейронная сеть с одним скрытым слоем может быть полностью характеризована зонотопом, который служит строительным блоком для более глубоких сетей.
	
	В разделах 2 и 3 мы введем необходимую нам базу тропической алгебры и тропической геометрии. Мы точно определим предположения в разделе 4 и установим связь между тропической геометрией и многослойной нейронной сетью в разделе 5. Мы проанализируем нейронные сети с помощью тропических методов в разделе 6, доказывая, что более глубокая нейросеть экспоненциально лучше выражает, чем менее глубокая --- хотя наша цель не столько показать результат анализа, сколько продемонстрировать, как тропическая геометрия может помочь получить полезные знания. Все доказательства отложены в раздел D дополнения.
	
	\section{Тропическая геометрия нейронных сетей}
	
	В разделе 5 нейронные сети определяются с помощью тропической алгебры, что позволяет нам изучать их с помощью тропической алгебраической геометрии. Мы покажем, что граница принятий решений нейронной сети ~--- это подмножество тропической гиперповерхности, соответствующего тропического полинома(Раздел 6.1). Мы увидим, что в некотором смысле, зонотопы образуют геомтрические строительные блоки для нейронных сетей (Раздел 6.2). Затем мы докажем, что геометрия функции, представленной нейронной сетью, становится значиттельно более сложной с увеличением ее количества слоёв.
	\subsection{Границы решений нейронной сети}
	
	Мы будем использовать тропическую геометрию и идеи из Раздела 5 для изучения границ решений нейронных сетей, фокусируясь на случае классификации двух категорий для ясности. Как объяснено в Разделе 4, нейронная сеть $\nu : \mathbf{R}^d \rightarrow \mathbf{R}^p$ вместе с выбором функции оценки $s: \mathbf{R}^p \rightarrow \mathbf{R}$ дают нам классификатор. 

\end{document}